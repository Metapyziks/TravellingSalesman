\documentclass[a4paper,11pt]{article}

\usepackage{mathtools}
\usepackage{lipsum}
\usepackage[margin=20mm]{geometry}
\usepackage{tabu}
\usepackage{wrapfig}
\usepackage{array,ragged2e,pst-node,pst-dbicons}
\usepackage{graphicx}

\title{Software Applications AI Search Assignment}
\date{\today}
\author{James King}

\begin{document}
\maketitle

\begin{abstract}
\emph{The Travelling Salesman Problem is an example of the NP-Complete class of
problems; attempting to find the optimal solution by trying every possibility
would require an obscene amount of time for all but the smallest of input
sizes. This report describes two methods for finding decent solutions in a 
reasonable amount of time. The first is a type of stochastic gradient descent,
and the second is a nature inspired algorithm that models a simple ant colony.
A piece of analysis software was also produced which attempts to display a
given graph in a form that is understandable to humans.}
\end{abstract}

\section{General Data Structures}
Each city graph, after being loaded from the input file, is stored in a 2D
integer array in such a way that the value in column $i$ and row $j$ is the
distance between city $i$ and city $j$. Representing the graph in this form
means that checking the distance between two cities is a simple and fast
operation, which was intended to make any search algorithms more time
efficient.

Tours are represented as fixed length one dimensional integer arrays, with
each value in the array being the index of the associated city. Each tour also
has two numbers stored within it that record the number of cities in the tour
and the total cost of the tour (the sum of the distances between each city in
the tour). The value for the tour cost is initially set internally to $-1$.
When the length of the tour is requested, if this value is $-1$ the cost is
calculated and stored in the place of the $-1$, ready to be returned again
without having to redundantly perform the costly calculation more than once. 
Additionally, any operations that could change the length of the tour will set
this value back to $-1$, signalling that the value needs to be recalculated
when next requested.

\begin{wrapfigure}{r}{0.38\textwidth}
\includegraphics[width=0.36\textwidth]{175vis}
\caption{The 175 city graph}
\end{wrapfigure}

When an empty tour is first created, the values in its integer array are the
indexes of each city in ascending order. When the $i$th city is added to the
tour, the item in the array with value equal to the city's index is swapped
with the value that was at the $i$th position. This means that for a tour of
length $j$, all values in the array from $j+1$ onwards are the indexes of
cities that are not currently in the tour and therefore can be selected to be
added to the tour. It is also possible to select the $k$th best city from this
subset, using the end of the array as a buffer which is partially ordered until
the city that is $k$th best using a given comparison criteria is found.

The graphs are given in a form that is almost impossible for a human to
understand and visualise, which can make it difficult to see what a search
method is doing right or wrong. I wrote a simple graph visualiser that attempts
to find a graphical representation of a given city file as a collection of
connected points on a euclidean plane. This is achieved by attaching simulated
dampened springs between each city with a strength proportional to the desired
distance between the two cities. This produced far better images than I had
anticipated; I was expecting all of the given graphs to have edge lengths that
could in no way be realistically approximated as euclidean distances. While
this is true for six of the graphs, the graphs with 12, 58, 127 and 535 cities
all have fairly nice visual representations. For example, the 175 city graph
(Figure 1) appears to consist of many rows of cities with a few others dotted
around them. The visualiser also has the ability to manipulate the positions of
cities and display a given tour, which can be modified also.

\section{Algorithm A}
My first algorithm is a combined stochastic gradient descent and constructive
search. The algorithm starts with an empty tour. For each step, a random city
from the set of unvisited cities is added to the end of the tour. Next, a
hill climbing algorithm operates on the tour until no new improvements are
found. This process repeats, adding a new city every step, until the tour is
complete. Then this tour's length is calculated and compared with the shortest
found so far. If this is a new shortest tour, it is recorded and the whole
algorithm starts again.

The hill climbing method used in this algorithm is based around reversing
sections of the tour. To perform the reversal between two cities $T_i$ and
$T_j$ (where $T_i$ is the $i$th city in tour $T$), first the positions of $T_i$
and $T_j$ are swapped in the tour. Then $i$ is incremented by $1$, and $j$ is
decreased by $1$. The swapping and incrementing / decrementing process repeats
until $i \ge j$, which is when all cities between the initial two selected ones
have been reversed in place.

\begin{figure}[h!]
\centering
\parbox{0.8\textwidth}{
\begin{center}
\begin{tabular}{c}
\Large
\Circlenode[radius=4mm,linestyle=dashed]{f1start}{~}
\hskip 5mm
\Circlenode[radius=4mm]{f1A}{A} \ncline[linestyle=dashed]{f1start}{f1A}
\hskip 5mm
\Circlenode[radius=4mm]{f1B}{B} \ncline{f1A}{f1B}
\hskip 5mm
\Circlenode[radius=4mm]{f1C}{C} \ncline{f1B}{f1C}
\hskip 5mm
\Circlenode[radius=4mm]{f1D}{D} \ncline{f1C}{f1D}
\hskip 5mm
\Circlenode[radius=4mm]{f1E}{E} \ncline{f1D}{f1E}
\hskip 5mm
\Circlenode[radius=4mm]{f1F}{F} \ncline{f1E}{f1F}
\hskip 5mm
\Circlenode[radius=4mm]{f1G}{G} \ncline{f1F}{f1G}
\hskip 5mm
\Circlenode[radius=4mm,linestyle=dashed]{f1end}{~}
\ncline[linestyle=dashed]{f1G}{f1end}
\ncarc[arcangle=30,linestyle=dashed]{-}{f1B}{f1F}
\ncarc[arcangle=30,linestyle=dashed]{-}{f1E}{f1A}
\\[1cm]
\Large
\Circlenode[radius=4mm,linestyle=dashed]{f2start}{~}
\hskip 5mm
\Circlenode[radius=4mm]{f2A}{A} \ncline[linestyle=dashed]{f2start}{f2A}
\hskip 5mm
\Circlenode[radius=4mm]{f2E}{E} \ncline{f2A}{f2E}
\hskip 5mm
\Circlenode[radius=4mm]{f2D}{D} \ncline{f2E}{f2D}
\hskip 5mm
\Circlenode[radius=4mm]{f2C}{C} \ncline{f2D}{f2C}
\hskip 5mm
\Circlenode[radius=4mm]{f2B}{B} \ncline{f2C}{f2B}
\hskip 5mm
\Circlenode[radius=4mm]{f2F}{F} \ncline{f2B}{f2F}
\hskip 5mm
\Circlenode[radius=4mm]{f2G}{G} \ncline{f2F}{f2G}
\hskip 5mm
\Circlenode[radius=4mm,linestyle=dashed]{f2end}{~}
\ncline[linestyle=dashed]{f2G}{f2end}

\end{tabular}
\caption{Segment of a tour before and after a reversal is applied between
	\emph{B} and \emph{E}, with the two new connections to be added represented
	by dashed arcs.}
\end{center}
}
\end{figure}
\vspace{-20pt}

The two cities to reverse between are chosen by checking every pair of cities
for the pair that would give the largest improvement to tour length if
reversed. The operation to find the change in tour length is made simple by the
fact that only two connections will be changed in a way that could affect the
tour length - the connections between the cities inside the swapped section are
only reversed but their length stays the same. The change in tour length is
therefore found like this:

$$(|\overrightarrow{{T_{i-1}}{T_i}}|
+ |\overrightarrow{{T_j}{T_{j+1}}}|)
- (|\overrightarrow{{T_{i-1}}{T_j}}|
+ |\overrightarrow{{T_i}{T_{j+1}}}|)$$

\noindent
Here $|\overrightarrow{{T_i}{T_j}}|$ is the distance between city $T_i$ and
$T_j$ as described by the city graph given as input.

Because of the nature of the reversal, some sets of pairs of cities would 
produce the same change in tour length when reversed. Pairs that would have
the same effect on tour length as others can be safely excluded to improve 
performance. I achieve this by picking the index $i$ of the first city, and the 
number of cities to reverse $c$, where $c \ge 2$ and
$c \le \frac{toursize}{2}$. This excludes swapping a city with itself (which
will never change the tour length) and any reversals that would return the same 
result as a previous one but with the whole tour in reverse.

The tour construction and reversal process is very fast, and because the
process is independent of any other systems related to the generation process
it may be easily threaded. This algorithm relies on being executed many, many
times to increase the probability of finding the optimal tour, and through
multithreading and other more general optimising techniques I improved the
speed of the algorithm to run thousands of tour tests a second. While the
algorithm may not be the smartest, by running it long enough it produced some
pretty small tour sizes. The algorithm keeps trying out new tours until no
better tours are found for a given number of attempts.

Initially the algorithm would only run the reversal improvement method after
the entire random tour was generated. This produced better results less
frequently and was actually slower despite running the improvement algorithm
only once per tour. The speed increase when improving after every new city
added is probably because the tour would require less reversal operations to
find a local minimum when the tour is larger (and the reversal algorithm is more
expensive) if the tour is constantly being maintained as it is built. However,
I am not certain that it is possible for the absolute optimal tour to be found
in every case when the tour is being improved as it is built whereas it is
trivial to see that finding the optimal tour is possible when only improving
after a random tour is built. In the end I decided to trade the possibility of
finding the optimal tour for finding sub optimal but decent tours more
frequently.

\section{Algorithm B}
My second algorithm is a constructive search using an Ant Colony heuristic.
The method places thousands of virtual ants at different starting positions on
the graph, which then wander between cities they have yet to visit until they
complete a tour. When a tour is complete, virtual pheromones are placed along
the route the ant took which influence other ants to take the paths along that
tour more often. Then the ant forgets which cities it has visited and starts
another tour.

The ants choose their next city to visit by first scoring each unvisited city
$C$ with the given criteria:

$$score = PW \times PS(C)+(1.0 - PW) \times \frac{(CT + 1.0)}{PL(C)}$$

Here $PW$ is a random number between $0.0$ and $1.0$,
$PS(C)$ is the potency of the pheromones left on the path between
the ant's current city and city $C$, $CT$ is the number of tours
completed by any ant which was a new shortest tour, and $PL(C)$ is the
length between the ant's current city and city $C$.

The equation is split into two parts, the first to score based on pheromone
level, and the second on path cost. The value of $PW$ determines how much the
score is weighted by pheromones, which was introduced so that some ants would
select cities on path cost while others would use pheromone potency.

After ranking each city in order by score, there is a random chance that the
ant will skip the best city and use the second best, and an even smaller chance
it will use the third best, and so on.

The amount of pheromone left when an ant completes a tour is given by:

$$pheromone = \frac{GraphCount}{TourCost - Minimum}$$

Here $GraphCount$ is the number of cities in the problem, $TourCost$ is the
total length of the tour, and $Minimum$ is the sum of the smallest cost from
each city to any neighbour. The total amount of pheromone between two cities is
also limited to being at most the number of fully completed tours that were new
shortest tours. The subtraction of $Minimum$ was added later to make improved
tour lengths have a bigger difference in pheromone potency. Since $TourCost$
can never be lower than $Minimum$, not subtracting this value would mean that
an improvement in tour length would lead to only an insignificant difference in
the denominator for larger graphs.

The equations used to determine which cities to visit were formulated using the
graph visualiser I had previously written; initially the ants would tend to
stop improving the route found after only a few iterations, so I modified the
graph visualiser to show the paths each ant was taking at each step. This
allowed me to repeatedly modify my path selection and pheromone placement
algorithms to try and force the ants to experience more routes, as tested by 
viewing them with the visualiser.

I noticed that eventually the simulation would stabilize without finding any
new tours for many steps, so I added a random probability for the pheromones
to be cleared if no better result was found in a while. This would give the
simulation the opportunity to find other tours.

Like my first algorithm, this method can easily be threaded to improve speed.
I also improved execution speed by recording which cities each ant had visited
in its current tour in an array of boolean values so that looking up if a tour
was unvisited would have a constant time cost instead of a linear one.

Admittedly I didn't spend as much time adjusting this algorithm as I did with
the first, mainly because the first algorithm produced very good results from
the start with relatively little effort. However, I expect that the ant colony,
given enough tweaking and thought, will be more effective at working towards
smaller tours.

\section{Results Analysis}
\begin{wraptable}{r}{0.5\textwidth}
\begin{center}
\begin{tabular}{| r | r  r  r |}
\hline
\textbf{City File} & \textbf{Result A} & \textbf{Result B} & \textbf{Difference} \\
\hline
\hline
SAfile012 & 56 & 56 & 0 (0.00\%) \\
SAfile017 & 1444 & 1444 & 0 (0.00\%) \\
SAfile021 & 2549 & 2549 & 0 (0.00\%) \\
SAfile026 & 1473 & 1473 & 0 (0.00\%) \\
SAfile042 & 1187 & 1188 & 1 (0.08\%) \\
SAfile048 & 12166 & 12505 & 339 (2.79\%) \\
SAfile058 & 25395 & 26027 & 632 (2.49\%) \\
SAfile175 & 21408 & 21774 & 366 (1.71\%) \\
SAfile180 & 1950 & 1950 & 0 (0.00\%) \\
SAfile535 & 48533 & 49061 & 528 (1.09\%) \\
\hline
\end{tabular}

\caption{Final results of both algorithms for each given city, and the
	difference between them.}
\end{center}
\end{wraptable}

The results displayed in Table 1 are the all-time lowest tour lengths
achieved by each algorithm. The table also shows the difference between the
results of the two algorithms, and the difference as a percentage of the
shortest tour's length when the difference is greater than zero.

The first obvious thing to note is that both algorithms achieved the exact same
tour length for the first four tours. Given the small size of the graphs,
I would be tempted to claim that this may be because the length found is the
absolute minimum. Both algorithms find their respective minimum tours in very
little time, and it is hard to differentiate the times taken by the two
algorithms.

Tours lengths for the graphs of size 42 and 48 differ very little between
the two methods, and the 58 city graph again has equal tour costs for both.
This may have something to do with the nature of that graph, which when
visualised shows that the paths have costs similar to euclidean distances, and
a clean tour can be traced around the outside with only a few high density
areas where the best route is ambiguous.

The next graph, with 175 cities, has a sudden jump in difference between the
two methods. When visualising the best tour produced by each algorithm it is
clear that the first method traces a neat route around the graph with no
unnecessary paths between distant cities. The second method's visualisation,
however, shows several wasteful jumps between the rows of cities.

The 180 node graph is an exceptional one. It has many paths between cities with
zero cost, which are probably designed to act like ``dead ends'' in that a
Best First Search will attempt to follow these and end up at a city with the
only paths available having extremely high cost. Both of my methods seemed to
avoid this issue. An interesting difference between the two algorithms
is that the first took a few hours of execution over the course of many runs to 
arrive at its minimum, and yet the second used to find the same result within a
few seconds until I implemented some optimisations to improve the result on
other graphs. Now the second algorithm rarely produces anything lower than
double the minimum it previously found.

The last (and largest) graph causes both algorithms to struggle. The first
takes noticeably longer to test each graph, and so can get through a smaller
sample in a given amount of time. However, it will still generate a better
result than a straight Best First Search or the second algorithm with its first
tests. The second algorithm produces its best result in a much longer time than
other graphs too. The difference in speed of finding minimal results is due in
part to the increased amount of time needed to perform operations on the larger
graph, and also because there are simply many more possible tours so the
probability of an arbitrary tour being minimal is substantially lower. Looking
at the visualisations, the first method navigates its way cleanly through the
regular grid-like clusters on one side of the graph, whereas the second mostly
traverses them in no real pattern.

When comparing simply by result score it is apparent that Algorithm A currently
can produce better results than Algorithm B. However, there isn't really much
scope for improvement with the first algorithm apart from speed optimisation
and maybe eliminating redundant tour checks. With Algorithm B, however, there
is plenty of opportunity to fine tune the method to attempt to get better
results, including the ability to alter it with optimisations for individual
graphs. Given more time, I am sure that it would be the better algorithm of the
two.
\end{document}
