\documentclass[a4paper,11pt]{article}

\usepackage{lipsum}
\usepackage{fullpage}
\usepackage{tabu}
\usepackage{array,ragged2e,pst-node,pst-dbicons}

\title{Computer System II Networks Assignment}
\date{January 16, 2013}
\author{James King}

\begin{document}
\maketitle

\begin{abstract}
\emph{The Travelling Salesman Problem is an example of the NP-Complete class of
problems; attempting to find the optimal solution by trying every possibility
would require an obscene amount of time for all but the smallest of input
sizes. This report describes two methods for finding decent solutions in a 
reasonable amount of time. The first is a type of stochastic gradient descent,
and the second is a nature inspired algorithm that models a simple ant colony.
A piece of analysis software was also produced to try and display each given
input in a form that is understandable to humans.}
\end{abstract}

\section{Algorithm A}
Initially I wanted to create a simple deterministic improvement search
algorithm to use as a benchmark when testing the capabilities of more complex
algorithms. Not wanting to simply copy a method I found from the internet, I
attempted to find my own solution without researching existing ones.

Each improvement search algorithm must have at least one mutation method which
can be applied to a route. It should be possible, though multiple applications
of this mutation method, to mutate any valid route into any other valid route.
A gradient descent algorithm is a variant of the improvement search which will
only perform mutations that immediately improve the current route, and this is
the class of algorithm I intended to design.

\subsection{Reversal Gradient Descent}
My first challenge was to decide on a mutation method. The method would need to
be quite simple in order to test many different mutations to find which ones
improve the tour. I opted for a mutator which selects two cities in the tour
and reverses the ordering of all cities between them. This method satisfies the
condition of being able to traverse the search space between any two tours,
given that swapping pairs of cities in the tour allows for any permutation of
tour to be explored. A swap can be performed through two reversals, the first
to swap the pair of cities you wish to exchange, and the second to reverse the
cities between then into their initial order. With this reversal mutation
method, it is also computationally simple to find how the length of the tour
will be affected. Because the connections between each pair of cities inside
the group being reversed will have the same cost (assuming the graph is
symmetric), only the two connections between the set of cities being reversed
and the rest of the tour will change in a way that could alter the tour length.

\begin{figure}
\begin{center}
\begin{tabular}{c}
\Large
\Circlenode[radius=4mm,linestyle=dashed]{f1start}{~}
\hskip 5mm
\Circlenode[radius=4mm]{f1A}{A} \ncline[linestyle=dashed]{f1start}{f1A}
\hskip 5mm
\Circlenode[radius=4mm]{f1B}{B} \ncline{f1A}{f1B}
\hskip 5mm
\Circlenode[radius=4mm]{f1C}{C} \ncline{f1B}{f1C}
\hskip 5mm
\Circlenode[radius=4mm]{f1D}{D} \ncline{f1C}{f1D}
\hskip 5mm
\Circlenode[radius=4mm]{f1E}{E} \ncline{f1D}{f1E}
\hskip 5mm
\Circlenode[radius=4mm]{f1F}{F} \ncline{f1E}{f1F}
\hskip 5mm
\Circlenode[radius=4mm]{f1G}{G} \ncline{f1F}{f1G}
\hskip 5mm
\Circlenode[radius=4mm,linestyle=dashed]{f1end}{~}
\ncline[linestyle=dashed]{f1G}{f1end}
\ncarc[arcangle=40,linestyle=dashed]{->}{f1B}{f1F}
\ncarc[arcangle=40,linestyle=dashed]{<-}{f1E}{f1A}
\\[2cm]
\Large
\Circlenode[radius=4mm,linestyle=dashed]{f2start}{~}
\hskip 5mm
\Circlenode[radius=4mm]{f2A}{A} \ncline[linestyle=dashed]{f2start}{f2A}
\hskip 5mm
\Circlenode[radius=4mm]{f2E}{E} \ncline{f2A}{f2E}
\hskip 5mm
\Circlenode[radius=4mm]{f2D}{D} \ncline{f2E}{f2D}
\hskip 5mm
\Circlenode[radius=4mm]{f2C}{C} \ncline{f2D}{f2C}
\hskip 5mm
\Circlenode[radius=4mm]{f2B}{B} \ncline{f2C}{f2B}
\hskip 5mm
\Circlenode[radius=4mm]{f2F}{F} \ncline{f2B}{f2F}
\hskip 5mm
\Circlenode[radius=4mm]{f2G}{G} \ncline{f2F}{f2G}
\hskip 5mm
\Circlenode[radius=4mm,linestyle=dashed]{f2end}{~}
\ncline[linestyle=dashed]{f2G}{f2end}
\end{tabular}
\end{center}
\caption{Segment of a tour before and after a reversal is applied between
	\emph{B} and \emph{E}, with the two new connections to be added represented
	by dashed arcs. The only connections that differ in cost to the original
	tour are the new connections from \emph{A} to \emph{E} and \emph{B} to
	\emph{F}.}
\end{figure}

\section{Results Analysis}
The results displayed in \emph{Table 1} are the all-time lowest tour lengths
achieved by each algorithm. The table also shows the difference between the
results of the two algorithms, and the difference as a percentage of the
shortest tour's length.

\begin{table}[h!]
\begin{center}
\begin{tabular}{| r | r  r  r |}
\hline
\textbf{City File} & \textbf{Result A} & \textbf{Result B} & \textbf{Difference} \\
\hline
\hline
SAfile012 & 56 & 56 & 0 (0.00\%) \\
SAfile017 & 1444 & 1444 & 0 (0.00\%) \\
SAfile021 & 2549 & 2549 & 0 (0.00\%) \\
SAfile026 & 1473 & 1473 & 0 (0.00\%) \\
SAfile042 & 1187 & 1188 & 1 (0.08\%) \\
SAfile048 & 12166 & 12505 & 339 (2.79\%) \\
SAfile058 & 25395 & 26027 & 632 (2.49\%) \\
SAfile175 & 21408 & 21774 & 366 (1.71\%) \\
SAfile180 & 1950 & 1950 & 0 (0.00\%) \\
SAfile535 & 48533 & 49061 & 528 (1.09\%) \\
\hline
\end{tabular}

\end{center}
\caption{Final results of both algorithms for each given city, and the
	difference between them.}
\end{table}

\lipsum[1-5]
\end{document}
